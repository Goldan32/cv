%!TEX TS-program = xelatex
%!TEX encoding = UTF-8 Unicode
% Awesome CV LaTeX Template
%
% This template has been downloaded from:
% https://github.com/posquit0/Awesome-CV
%
% Author:
% Claud D. Park <posquit0.bj@gmail.com>
% http://www.posquit0.com
%
% Template license:
% CC BY-SA 4.0 (https://creativecommons.org/licenses/by-sa/4.0/)
%


%%%%%%%%%%%%%%%%%%%%%%%%%%%%%%%%%%%%%%
%     Configuration
%%%%%%%%%%%%%%%%%%%%%%%%%%%%%%%%%%%%%%
%%% Themes: Awesome-CV
\documentclass[]{awesome-cv}
\usepackage{textcomp}
%%% Override a directory location for fonts(default: 'fonts/')
\fontdir[fonts/]

%%% Configure a directory location for sections
\newcommand*{\sectiondir}{resume/}

%%% Override color
% Awesome Colors: awesome-emerald, awesome-skyblue, awesome-red, awesome-pink, awesome-orange
%                 awesome-nephritis, awesome-concrete, awesome-darknight
%% Color for highlight
% Define your custom color if you don't like awesome colors
\colorlet{awesome}{awesome-red}
%\definecolor{awesome}{HTML}{CA63A8}
%% Colors for text
%\definecolor{darktext}{HTML}{414141}
%\definecolor{text}{HTML}{414141}
%\definecolor{graytext}{HTML}{414141}
%\definecolor{lighttext}{HTML}{414141}

%%% Override a separator for social informations in header(default: ' | ')
%\headersocialsep[\quad\textbar\quad]
    \begin{document}
    
%%%%%%%%%%%%%%%%%%%%%%%%%%%%%%%%%%%%%%
%     Profile
%%%%%%%%%%%%%%%%%%%%%%%%%%%%%%%%%%%%%%
\begin{center}
	\headerfirstnamestyle{Dániel} \headerlastnamestyle{Arany} \\
	\vspace{2mm}
	{\faEnvelope\ arany.daniel1999@gmail.com} | {\faMobile\ +36 30 290 1219} | {\faLink\ https://github.com/Goldan32}
\end{center}
%%%%%%%%%%%%%%%%%%%%%%%%%%%%%%%%%%%%%%
%     Education
%%%%%%%%%%%%%%%%%%%%%%%%%%%%%%%%%%%%%%
\cvsection{Education}
\begin{cventries}
	\cventry
	{MSc in Electrical Engineering}
	{Budapest University of Technology and Economics}
	{Budapest}
	{February 2022 – Present}
	{}
	\cventry
	{BSc in Electrical Engineering}
	{Budapest University of Technology and Economics}
	{Budapest}
	{September 2018 – January 2022}
	{GPA: 4.6}
\end{cventries}

\vspace{-2mm}
%%%%%%%%%%%%%%%%%%%%%%%%%%%%%%%%%%%%%%
%     Experience
%%%%%%%%%%%%%%%%%%%%%%%%%%%%%%%%%%%%%%
\cvsection{Work Experience}
\begin{cventries}
	\cventry
	{Engineer, Junior Software}
	{Flex}
	{}
	{2022 – Present}
	{\begin{cvitems}
		\item {Implementing requested features in a bigger project using C/C++, BASH and yocto recipies.}
		\item {Suggesting and implementing a firmware component version reader tool in C.}
		\end{cvitems}}
	\cventry
	{Software Developer Trainee}
	{Flex}
	{}
	{2021 – 2022}
	{\begin{cvitems}
		\item {Assisting the Firmware Team by writing low level software codes and unit tests.}
		\item {Implementing bigger features and writing documentation.}
		\end{cvitems}}
	\cventry
	{Demonstrator}
	{Budapest University of Technology and Economics}
	{}
	{}
	{\begin{cvitems}
		\item {Teaching intro level programming and digital technology classes for first year students, by explaining and presenting solutions to tasks.}
		\item {Assisting the instructor in computer labs by helping the students individually.}
		\end{cvitems}}
	\cventry
	{Student Council Representative}
	{Budapest University of Technology and Economics}
	{}
	{2020 – 2021}
	{\begin{cvitems}
		\item {Communicating with students mostly via email and advising them about school policy.}
		\item {Representing student interests at various meetings.}
		\end{cvitems}}
\end{cventries}
\cvsection{Skills}
\begin{cventries}
	\cventry
	{}
	{\def\arraystretch{1.15}{\begin{tabular}{ l l }
		Programming Languages:  & {\skill{ C/C++, Embedded C, Rust, Python, BASH Scripting, Matlab, Verilog, Robot Framework}} \\
		Development Enviroments:  & {\skill{ gcc/g++, Makefiles, CMake, Visual Studio, Eclipse based IDE-s, OpenBMC, Petalinux}} \\
		Other skills:  & {\skill{ Git, Linux (usage and development), Yocto Project, PCB Design (KiCAD), JIRA, Confluence}} \\
		\end{tabular}}}
	{}
	{}
	{}
\end{cventries}

\vspace{-7mm}
\cvsection{Projects}
\begin{cventries}
	\cventry
	{A secondary bootloader for Infineon Aurix microcontroller. Implementing Software Over The Air by receiving the new image via TFTP, loading it into memory and activating it.}
	{BSc Thesis}
	{C, Bash, Makefile}
	{}
	{}
	
	\vspace{-5mm}
	\cventry
	{A sensor network capable of monitoring temperature lightning and other similar qualities. Using a BeagleBone and an ESP32 connected together as the head of the network, and ESP8266-s with sensors as the nodes.}
	{Sensor Network with BeagleBone and ESP32}
	{C++, Python}
	{https://github.com/Goldan32/onlab}
	{}
	
	\vspace{-5mm}
\end{cventries}

\ 
\end{document}